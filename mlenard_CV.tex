% Michael Lenard - Curriculum Vitae
% Email: mclenard@umich.edu
% Web: https://mclenard.github.io/
% Repo: https://github.com/mclenard/cv

\documentclass[12pt,letterpaper]{report}

\usepackage[T1]{fontenc} % output T1 font encoding (8-bit) so accented characters are a single glyph
\usepackage[utf8]{inputenc} % allow input of utf-8 encoded characters
\usepackage[strict,autostyle]{csquotes} % smart and nestable quote marks
\usepackage[USenglish]{babel} % automatically regionalize hyphens, quote marks, etc
\usepackage{microtype} % improves text appearance with kerning, etc
\usepackage{datetime} % enable formatting of date output
\usepackage{tabto} % make nice tabbing
\usepackage{hyperref} % enable hyperlinks and pdf metadata
\usepackage{geometry} % manually set page margins
\usepackage{enumitem} % enumerate with [resume] option
\usepackage{titlesec} % allow custom section fonts

% what is your name?
\newcommand{\myname}{Michael C. Lenard}

% define a default font face and set it as the body font
\usepackage{crimson} % document's serif font
\usepackage{helvet}  % document's sans serif font

% define how far to tab for list items with left-aligned date - different font faces need different widths
\newcommand{\listtabwidth}{1.75cm}

% set name font to title the document
\newcommand{\namefont}[1]{{\normalfont\bfseries\Huge{#1}}}

% set section heading fonts and before/after spacing
\SetTracking{encoding=*}{20}
\titleformat{\section}{\sffamily\small\bfseries\lsstyle\uppercase}{}{}{}{}
\titlespacing{\section}{0pt}{24pt plus 4pt minus 2pt}{12pt plus 2pt minus 2pt}

% set subsection heading fonts and before/after spacing
\titleformat{\subsection}{\sffamily\footnotesize\bfseries}{}{}{}{}
\titlespacing{\subsection}{0pt}{12pt plus 4pt minus 2pt}{8pt plus 2pt minus 2pt}

% set page margins
\geometry{body={6.5in, 9.0in},
    left=1.0in,
    top=1.0in}

% prevent paragraph indentation
\setlength\parindent{0em}

% define space between list items
\newcommand{\listitemspace}{0.25em}

% make unordered lists without bullets and use compact spacing
\renewenvironment{itemize}
{\begin{list}{}{\setlength{\leftmargin}{0em}
            \setlength{\parskip}{0em}
            \setlength{\itemsep}{\listitemspace}
            \setlength{\parsep}{\listitemspace}}}
{\end{list}}

% make tabbed lists so content is left-aligned next to years
\TabPositions{\listtabwidth}
\newlist{tablist}{description}{3}
\setlist[tablist]{leftmargin=\listtabwidth,
    labelindent=0em,
    topsep=0em,
    partopsep=0em,
    itemsep=\listitemspace,
    parsep=\listitemspace,
    font=\normalfont}

% print the month and year only when using \today
\newdateformat{monthyeardate}{\monthname[\THEMONTH] \THEYEAR}

% define hyperlink appearance and metadata for pdf properties
\hypersetup{
    colorlinks  = true,
    urlcolor    = black,
    pdfauthor   = {\myname},
    pdfkeywords = {research data, LIS, information science, RDM},
    pdftitle    = {\myname: Curriculum Vitae},
    pdfsubject  = {Curriculum Vitae},
    pdfpagemode = UseNone
}

\begin{document}
 \raggedright

% display name as the document title
\namefont{\myname}

% contact info
\vspace{1em}
\begin{minipage}[t]{0.68\textwidth}
        2118 Commonwealth Dr. \#30 \\
Charlottesville, VA 22901
\end{minipage}
\begin{minipage}[t]{0.31\textwidth}
        Email: \href{mailto:lenard@virginia.edu}{lenard@virginia.edu} \\
        Phone: +1 734 899 0188
\end{minipage}
\vspace{0.5em}



\section*{Education}

\begin{tablist}

	\item[M.S.I.] \tab Information, University of Michigan, 2020

	\item[M.S.]  \tab Physical Chemistry, University of Michigan, 2017

	\item[B.A.]  \tab Political Science, Michigan State University, 2011
        
	\item[B.S.]  \tab Physics, Michigan State University, 2011

\end{tablist}



\section*{Professional Positions}

\begin{tablist}

\item[2023--] \tab Research Data Management Librarian, University of Virginia Library

\item[2020--22] \tab Project Manager, Thomer Lab, University of Michigan School of Information

\end{tablist}



\section*{Professional and Academic Interests}

\begin{itemize}

	\item \textit{Research data:} data curation and management, facilitating access to research data, data cleaning and processing, developing and documenting transparent data workflows for reproducible scientific analysis, developing best practices for data management during the research process

	\item \textit{Libraries' role in science:} outreach to and collaboration with scientific researchers, support for science throughout the research lifecycle, advocating for open science and equitable access to scientific research and resources, getting involved with citizen and community science projects

	\item \textit{Technical instruction:} improving pedagogy and elevating the quality of materials for technical skills instruction and data/scientific/information literacy instruction to create better, more effective guides, tutorials and workshops

	\item \textit{Semantics-based knowledge organization:} linked data, taxonomies, ontologies, thesauri, classification schemes, controlled vocabularies, and how they can be used to help structure and organize complex born-digital research outputs

\end{itemize}



\section*{Pre-professional Research \& Project Experience}

\begin{tablist}

	\item[2022] \tab Data Repository Assistant, University of Michigan Library

	\begin{itemize} \begin{footnotesize}

		\item - Assessed datasets being deposited into Deep Blue repository and consulted with faculty regarding how they could be better documented and made fit for archiving

	\end{footnotesize} \end{itemize}
    
	\item[2019--22] \tab Research Assistant, Thomer Lab, University of Michigan School of Information

	\begin{itemize} \begin{footnotesize}

		\item - Analyzed the structure and content of Throughput, an Earth Science graph database
		
		\item - Analyzed and summarized qualitative interview data for the Migrating Research Data Collections project

		\item - Cleaned and helped design processing workflows for natural history datasets from La Brea and the Michigan Institute for Fisheries Research

	\end{footnotesize} \end{itemize}
        
	\item[2018--20] \tab Resident \& Program Assistant, Shapiro Design Lab, University of Michigan Library

	\begin{itemize} \begin{footnotesize}

		\item - Created data processing workflows for the Lab’s Zooniverse projects 

		\item - Crosswalked Ann Arbor biodiversity data to Darwin Core for upload to GBIF

	\end{footnotesize} \end{itemize}

	\item[2020] \tab Summer Librarian, University of Michigan Biological Station

	\begin{itemize} \begin{footnotesize}

		\item - Provided reference service and completed collection development projects

	\end{footnotesize} \end{itemize}
    
	\item[2019] \tab Technical Services Project Assistant, University of Michigan Library
	
	\begin{itemize} \begin{footnotesize}
	
		\item - Assessed BIBFRAME record dataset for accuracy \& fitness for use by the Library
		
	\end{footnotesize} \end{itemize}

	\item[2015--17] \tab Research Assistant, Geva Group, University of Michigan Dept. of Chemistry
	
	\item[2013--14] \tab Research Assistant, Hoogstraten Laboratory, Michigan State University Dept. of Biochemistry and Molecular Biology

\end{tablist}



\section*{Instruction Experience}

\subsection*{Workshops (General)}

\begin{itemize}

	\item Library Carpentry workshop. (Co-instructor, Feb. 2022)
	
	\item Data Carpentry workshop: Ecology with R. (Co-instructor, Nov. 2021)
	
	\item Software Carpentry workshop. (Helper, May 2022)

\end{itemize}

\subsection*{Workshops (UVA)}

\begin{itemize}

	\item Reproducible Analysis and Documentation with R and R Markdown/Quarto

	\item Preparing Datasets for Publishing

\end{itemize}

\subsection*{University of Michigan}
    
\begin{itemize}
    	
    	\item SI 666: Organization of Information Resources (Graduate Student Instructor)
        
	\item SI 106: Programs, Information, and People (Graduate Student Instructor)
        
	\item SI 699: Digital Curation Mastery Course (Tutor)
        
	\item CHEM 125/126: General Chemistry Laboratory I \& II (Graduate Student Instructor)
        
	\item CHEM 453: Biophysical Chemistry I (Graduate Student Instructor)
        
	\item CHEM 260/261: Chemical Principles \& Introduction to Quantum Chemistry (Graduate Student Instructor)
        
	\item CHEM 230: Physical Chemical Principles and Applications (Graduate Student Instructor)
    
\end{itemize}
    	
\subsection*{Michigan State University}

\begin{itemize}

        \item PHY 184: Physics for Scientists and Engineers II (Teaching Assistant)
        
        \item PHY 232: Introductory Physics II (Teaching Assistant)
        
        \item PHY 191: Physics Laboratory for Scientists I (Teaching Assistant)
        
        \item PHY 251: Introductory Physics Laboratory I (Teaching Assistant)
        
        \item PHY 102: Physics Computations I (Teaching Assistant)

\end{itemize}



\section*{Publications}

\subsection*{Articles}

\begin{tablist}
   
        \item[2022] \tab Thomer, A. K., Starks, J. R., Rayburn, A., \& \textbf{Lenard, M.} Maintaining Repositories, Databases, and Digital Collections in Memory Institutions: An Integrative Review. \textit{Proceedings of the Association for Information Science and Technology 59}: 310-323. \href{https://doi.org/10.1002/pra2.755}{https://doi.org/10.1002/pra2.755}
        
        \item[2017] \tab Jafari, M., Welden, A., Williams, K., Winograd, B., Hendrickson, H., \textbf{Lenard, M.}, Gottfried, A., \& Geva, E. Compute-to-Learn: Authentic Learning via Development of Interactive Computer Demonstrations within a Peer-Led Studio Environment. \textit{J. Chem. Ed. 94}(12): 1896-1903. \href{https://pubs.acs.org/doi/10.1021/acs.jchemed.7b00032\#}{https://pubs.acs.org/doi/10.1021/acs.jchemed.7b00032\#}

        \item[2015] \tab Wiley, T., Arruda, B., Miller, N., \textbf{Lenard, M.}, \& Sension, R. Excited electronic states and internal conversion in cyanocobalamin. \textit{CCL 26}(4): 439-444. \href{https://doi.org/10.1016/j.cclet.2015.03.003}{https://doi.org/10.1016/j.cclet.2015.03.003}

\end{tablist}

\subsection*{Book Chapters}

\begin{tablist}

	\item[2023] \tab Thomer, A. K., Wofford, M. F., \textbf{Lenard, M.}, Dominguez Vidana, S. E., \& Goring, S. J. Revealing Earth Science code and data use practices using the Throughput Graph Database. In Ma, X., Mookerjee, M., Hsu, L. \& Hills, D. (eds.), Recent Advancement in Geoinformatics and Data Science: Geological Society 
of America Special Paper 558. \href{https://doi.org/10.1130/2022.2558(10)}{https://doi.org/10.1130/2022.2558(10)}
	
\end{tablist}

\subsection*{White Papers}

\begin{tablist}

	\item[2022] \tab \textbf{Lenard, M.}, Thomer, A. K. Draft of Statistical Metadata Standards--In Detail. In \textit{Transparency in Statistical Information for the National Center for Science and Engineering Statistics and All Federal Statistical Agencies.} 177-218. Washington, DC: The National Academies Press. \href{https://doi.org/10.17226/26360}{https://doi.org/10.17226/26360.}
	
\end{tablist}



\section*{Presentations}

\subsection*{Talks}

\begin{tablist}

	\item[2021] \tab Dominguez Vidana, S., Goring, S. J., \textbf{Lenard, M.}, Wofford, M., \& Thomer, A. K. (2021, October 13). Machine Learning in the Earth Sciences: A Broad Survey with Use Cases from the Throughput Database. GSA Connects, Portland, OR. \href{https://doi.org/10.1130/abs/2021AM-370665}{https://doi.org/10.1130/abs/2021AM-370665}
	    
%	\item[2021] \tab Dominguez Vidana, S., Thomer, A. K., Wofford, M., \textbf{Lenard, M.}, \& Goring, S. J. (2021, December 13--17). Discovering and Describing links between notebooks and analysis in code repositories, data, and publications in the Earth Sciences. AGU Fall Meeting, New Orleans, LA.

\end{tablist}
    
\subsection*{Posters}
    
\begin{tablist}
    
	\item[2022] \tab Starks, J., \textbf{Lenard, M.}, \& Thomer, A. K. (2022, March 15--17). What best practices exist to support database and digital collection migration? RDAP (Virtual). \href{https://osf.io/vxe3a/}{https://doi.org/10.17605/OSF.IO/VXE3A}
	
	\item[2021] \tab Wofford, M. F., Goring, S. J., \textbf{Lenard, M.}, Dominguez Vidana, S. E., \& Thomer, A. K. (2021, December 9). Discovering data reuse with the Throughput Annotation Database. FORCE 2021 Online Conference. \href{https://doi.org/10.5281/zenodo.5768578}{https://doi.org/10.5281/zenodo.5768578}
	
	\item[2020] \tab \textbf{Lenard, M.} (2020, February). Zooniverse Data Workflows. UMSI QuasiCon, Ann Arbor, MI.

\end{tablist}



\section*{Skills and Other Experience}

\begin{itemize}

	\item Knowledge of scientific literature and journal databases, the scientific research process, scientific research practices, and the research data lifecycle
        
	\item Experience with Darwin Core, Dublin Core, MARC, METS, OAIS, BIBFRAME, XML, JSON, RDF, RDFS, SKOS, OWL, PROV, and various other data and metadata standards
        
	\item Competency with Python, R, Regular expressions, SQL, SPARQL, OpenRefine, Protégé, Mathematica; familiarity with MATLAB
	
	\item Certified instructor for The Carpentries workshops
        
	\item Familiarity with statistical analyses and data cleaning and wrangling techniques

	\item Experience working with large, heterogeneous scientific databases and datasets in several domains

	\item Ability to learn new systems and software quickly, including \textit{inter alia} integrated library systems, stats packages, various APIs, content management systems, and database software
        
\end{itemize}



\section*{Awards, Honors, and Fellowships}

\begin{tablist}
    
	\item[2019] \tab Rackham Diversity, Equity, and Inclusion Certificate

	\item[2014--17] \tab Rackham Science Award (Fellowship)
        
	\item[2011--12] \tab Kaplan, Strangis and Kaplan Law School Scholarship 
        
	\item[2011--12] \tab Dean's Distinguished Scholarship
        
	\item[2010] \tab Herbert T. Graham Scholarship
        
	\item[2006--10] \tab Michigan Competitive Scholarship

\end{tablist}



% display today's date as Month Year after a vertical space below the end of the text
\begin{center}
 %\vspace{6em}
        \vfill
        Updated \monthyeardate\today
\end{center}


\end{document}
